\documentclass{scrartcl}

\usepackage[hidelinks]{hyperref}
\usepackage[none]{hyphenat}

\title{Essay Proposal}
\subtitle{COMP150 - Agile Essay}

\author{Jonathan Marx}

\begin{document}

\maketitle

\section*{Topic}

My essay will be on: "What are the key benefits and drawbacks of daily scrum meetings and how could they be leveraged or overcome in a game development project?". To answer this question I will look at how daily scrum meetings are currently used in agile development and some of the protentional weaknesses of the current method. I will also research into better practices of the daily scrum which could make the system more effective.

% Add details as appropriate.

\section*{Paper 1}
% This is an example! Replace the details with a paper relevant to your chosen topic.
\begin{description}
\item[Title:] Do Daily Scrums Have to Take Place Each Day?
\item[Citation:] \cite{shannon}
\item[Abstract:] Agile development approaches, such as Scrum, continue to gain importance in today's world. Since previous research has predominantly treated development approaches as a black box, we answer the call for empirical research concerning adoption of agile methods. The study's aim is to assess the adoption or adaption of Scrum principles at an e-commerce company. The findings of our in-depth single case study reveal that not all Scrum principles are suitable in each context. By discussing the reasons for adopting or adapting the principles, we contribute to a deeper understanding of agile methods and help to open the black box of development approaches in information systems research. The in-depth insights gained at our case company provide practitioners a useful reference for adapting agile methods to their specific contexts.
\item[Web link:] \url{http://ieeexplore.ieee.org.ezproxy.falmouth.ac.uk/document/7070423/}
\item[Full text link:] \url{http://ieeexplore.ieee.org.ezproxy.falmouth.ac.uk/stamp/stamp.jsp?arnumber=7070423}
\item[Comments:] This is a case study of customised scrum principles at an E-commerce Company. It discusses the idea of daily scrum meetings not having to take place every day. 
\end{description}

\section*{Paper 2}
\begin{description}
\item[Title:] Supporting daily scrum meetings with change structure
\item[Citation:] \cite{bibtex_key}
\item[Abstract:] A flexible cooperative task board for supporting daily scrum meetings is described as an application of different hypermedia domains. In addition, change structure is introduced as a means to explicitly model changes in task management. It helps the scrum development team in a sprint retrospective to improve their planning.
\item[Web link:] \url{http://dl.acm.org.ezproxy.falmouth.ac.uk/citation.cfm?id=1557927}
\item[Full text link:] \url{http://delivery.acm.org.ezproxy.falmouth.ac.uk/10.1145/1560000/1557927/p57-rubart.pdf?ip=193.61.64.8&id=1557927&acc=ACTIVE%20SERVICE&key=BF07A2EE685417C5%2EEAA225A8AB01C582%2E4D4702B0C3E38B35%2E4D4702B0C3E38B35&CFID=865857685&CFTOKEN=33760086&__acm__=1479394340_72e150c1a27f02d0b4d6c7b786d4f901}
\item[Comments:] A flexible cooperative task board for supporting daily scrum meetings is described as an application of different hypermedia domains. It helps the scrum development team in a sprint retrospective to improve their planning.
\end{description}

\section*{Paper 3}
\begin{description}
\item[Title:] Obstacles to Efficient Daily Meetings in Agile Development Projects
\item[Citation:] \cite{bibtex_key}
\item[Abstract:] Context: Most of the software organizations that use agile methods organize daily team meetings. Aim: Our aim was to understand how daily meetings are conducted and identify obstacles that reduce their efficiency. Method: We observed 56 daily meetings and conducted 21 interviews in three different teams in two countries. We used the repertory grid technique in the interviews and to analyze the results. Results: We identified thirteen obstacles. The four most prominent ones were: (1) The daily meetings lasted too long (on average, 22 minutes instead of the scheduled 15 minutes). (2) In the meetings that were not self-organized, team members reported to the Scrum Master instead of sharing information among all team members. (3) The interruption caused by daily meetings required substantially more time than the actual meeting time due to overhead before and after the meetings. (4) Several team members had negative attitudes towards the daily meetings. Conclusion: Organizers of daily meetings should evaluate whether the obstacles we have identified are present in their organization and consider our suggestions to remove or reduce these obstacles.
\item[Web link:] \url{http://ieeexplore.ieee.org.ezproxy.falmouth.ac.uk/document/6681342/}
\item[Full text link:] \url{http://ieeexplore.ieee.org.ezproxy.falmouth.ac.uk/stamp/stamp.jsp?arnumber=6681342}
\item[Comments:] The aim of this paper was to understand how daily meetings are conducted and identify obstacles that reduce their efficiency. They observed and analyses lots of daily meetings and identified the obstacles. 
\end{description}

\section*{Paper 4}
\begin{description}
\item[Title:] Daily scrums in a distributed environment
\item[Citation:] \cite{bibtex_key}
\item[Abstract:] Agile product teams tend to be collocated in an effort to facilitate communication and collaboration between members, but today, remote workforces are a reality. Agilists that are looking to use Scrum typically find advice that is relevant to collocated teams, but don't find much that deals with issues of remote teams. As we move more and more toward distributed teams we need to identify best practices for working with remote (perhaps culturally different) teammates. We need to look at our methods and the possibility of modifying our practices to better fit with multiple time zones, cultural differences and tools that need to go beyond "stories on the wall".
\item[Web link:] \url{http://dl.acm.org.ezproxy.falmouth.ac.uk/citation.cfm?id=1723101}
\item[Full text link:] \url{http://delivery.acm.org.ezproxy.falmouth.ac.uk/10.1145/1730000/1723101/p349-ganis.pdf?ip=193.61.64.8&id=1723101&acc=ACTIVE%20SERVICE&key=BF07A2EE685417C5%2EEAA225A8AB01C582%2E4D4702B0C3E38B35%2E4D4702B0C3E38B35&CFID=865857685&CFTOKEN=33760086&__acm__=1479394621_56f5fb57af25f1a79a6c49932b096ea5}
\item[Comments:] This paper looks at how Agile development teams can work in distributed environments. It looks at the possibility of modifying our practices to better fit with multiple time zones, cultural differences and tools that need to go beyond “stories on the wall”

\end{description}

\section*{Paper 5}
\begin{description}
\item[Title:] A Cooperative Multitouch Scrum Task Board for Synchronous Face-to-Face Collaboration
\item[Citation:] \cite{bibtex_key}
\item[Abstract:] The Scrum planning approach to software development is a widespread agile methodology. In a setting, which is ideal for Scrum, the team is collocated. Daily Scrum meetings support the team in organizing itself. Team members meet in front of a task board and update their status of work, originally based on paper-based notes. In this poster, we present a Multitouch Scrum Task Board, which is designed for being used by a team synchronously during face-to-face collaboration. Tasks can be created, explored, sorted, resized, and visually arranged in parallel simultaneously by multiple users on a multitouch display. We present our first experiences using the cooperative Scrum Task Board on a multitouch table.
\item[Web link:] \url{http://dl.acm.org.ezproxy.falmouth.ac.uk/citation.cfm?id=2669551}
\item[Full text link:] \url{http://dl.acm.org.ezproxy.falmouth.ac.uk/ft_gateway.cfm?id=2669551&ftid=1513795&dwn=1&CFID=865857685&CFTOKEN=33760086}
\item[Comments:] This paper introduces the “Multitouch Scrum Task Board” which is designed for being used by a team synchronously during face-to-face collaboration. Tasks can be created, explored, sorted, resized, and visually arranged in parallel simultaneously by multiple users on a multitouch display.
\end{description}

\section*{Paper 6}
\begin{description}
\item[Title:] Investigating Daily Team Meetings in Agile Software Projects
\item[Citation:] \cite{bibtex_key}
\item[Abstract:] An increasing amount of time is being spent at organizational meetings. One common type of meeting in software projects is the daily team meeting, which is the most important forum for coordinating and planning daily work. To better understand how software teams make decisions, communicate, and coordinate their work, we must uncover the micro-level interaction processes among the team members at these meetings. We analyzed transcriptions of eight daily meetings from two software development teams. The agile literature states that the daily meeting should focus on answering questions such as "What have I done? What will be done? What obstacles are in my way?" However, on average, only 24% of each of the meetings that we studied focused on this task. We found that 35% of the meeting was spent on elaborating problem issues and discussing possible solutions. Very little time was used for coordinating tasks. Our results indicate that many project decisions are made in daily team meetings and that this quick decision making requires team members to be experts. These experts need to have a shared understanding of who is responsible for what and of the information and requirements needed to solve the tasks.
\item[Web link:] \url{http://ieeexplore.ieee.org.ezproxy.falmouth.ac.uk/document/6328161/}
\item[Full text link:] \url{http://ieeexplore.ieee.org.ezproxy.falmouth.ac.uk/stamp/stamp.jsp?arnumber=6328161}
\item[Comments:] This paper uncovers the micro-level interaction processes among the team members at daily meetings to better understand how teams make decisions, communicate and coordinate their work.
\end{description}

\bibliographystyle{ieeetran}
\bibliography{initial_references}

\end{document}
